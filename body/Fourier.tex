\section{傅里叶变换的部分相关概念及性质}

\subsection{傅里叶级数}

\textit{我们知道,当一个函数$f(x)$满足} \textbf{迪利克雷(Dirichlet)条件} \textit{即在区间${[-\frac{T}{2},\frac{T}{2}]}$上满足} :

\qquad \qquad \text{1、f(x)在区间${[-\frac{T}{2},\frac{T}{2}]}$上连续或者只有有限个第一间断点;}

\qquad \qquad \text{2、f(x)在区间${[-\frac{T}{2},\frac{T}{2}]}$上只有有限个极值点。}

\textit{$f(x)$就可以展开成傅里叶级数} :
\[f(x)=\frac{a_0}{2}+\sum_{n=1}^{+\infty}(a_n\cos(n\omega x)+b_n\sin(n\omega x)) \tag{a}\]

\textit{其中:}
\[\omega=\frac{2\pi}{T} \qquad a_0=\frac{2}{T}\int_{-\frac{T}{2}}^{\frac{T}{2}}f(x)\dd{x}\]
\[a_n=\frac{2}{T}\int_{-\frac{T}{2}}^{\frac{T}{2}}f(x)\cos(n\omega x)\dd{x} \qquad b_n=\frac{2}{T}\int_{-\frac{T}{2}}^{\frac{T}{2}}f(x)\sin(n\omega x)\dd{x}\]
\[(n=1,2,3......) \tag{b}\]

\textit{将$\cos(n\omega x)$、$\sin(n\omega x)$写成复指数形式} :
\[\cos(n\omega x)=\frac{e^{in\omega x}+e^{-in\omega x}}{2} \qquad \sin(n\omega x)=\frac{e^{in\omega x}-e^{-in\omega x}}{2i} \tag{c}\]

\textit{则} (a) \textit{式可以写成以下形式 }:
\[f(x)=\frac{a_0}{2}+\sum_{n=1}^{+\infty}(\frac{a_n-ib_n}{2}e^{in\omega x}+\frac{a_n+ib_n}{2})e^{-in\omega x} \tag{d}\]

\textit{我们令 }:
\[c_0=\frac{a_0}{2}=\frac{1}{T}\int_{-\frac{T}{2}}^{\frac{T}{2}}f(x)\dd{x}\]
\[c_n=\frac{a_n-ib_n}{2}=\frac{1}{T}(\int_{-\frac{T}{2}}^{\frac{T}{2}}f(x)(\cos(n\omega x)\dd{x}-i\sin(n\omega x))\dd{x})=\frac{1}{T}\int_{-\frac{T}{2}}^{\frac{T}{2}}f(x)e^{-in\omega x}\dd{x}\]
\[c_{-n}=\frac{a_n+ib_n}{2}=\frac{1}{T}(\int_{-\frac{T}{2}}^{\frac{T}{2}}f(x)(\cos(n\omega x)\dd{x}+i\sin(n\omega x))\dd{x})=\frac{1}{T}\int_{-\frac{T}{2}}^{\frac{T}{2}}f(x)e^{in\omega x}\dd{x}\]
\[(n=1,2,3......)\]

\textit{合并写成 }:
\[c_n=\frac{1}{T}\int_{-\frac{T}{2}}^{\frac{T}{2}}f(x)e^{-in\omega x}\dd{x} \qquad (n=\pm 1,\pm 2,\pm 3......) \tag{e}\]

\textit{则}(a)\textit{式可以写成 }:
\[f(x)=c_0+\sum_{n=1}^{+\infty}(c_ne^{i\omega_n x}+c_{-n}e^{-i\omega_n x})=\sum_{n=-\infty}^{+\infty}c_ne^{i\omega_n x} \qquad (\omega_n:=n\omega) \tag{f}\]

\textit{最终,我们得到$f(x)$复指数形式的傅里叶级数 }:
\[f(x)=\sum_{n=-\infty}^{+\infty}(\frac{1}{T}\int_{-\frac{T}{2}}^{\frac{T}{2}}f(y)e^{-i\omega_n y}\dd{y})e^{i\omega_n x} \tag{g}\]

\subsection{傅里叶积分定理}

\textit{在实际的应用中,我们得到的信号曲线不一定是一个周期函数,那么我们有没有办法将一个非周期的函数展开成傅里叶级数呢} ?

\textit{对于一个定义在实数轴上的非周期的函数,我们可以认为它的周期是无穷大,对此我们将把能展开成傅里叶级数的周期函数延拓到整个实数轴上,再将其周期取为无穷大。}

\textit{将$f(x)$按周期$T$在实数域上做延拓的到函数$f_T(x)$ }:
\[ f_T(x) := \left\{
\begin{array}{rl}
f(x) & \text{if } x \in [-\frac{T}{2},\frac{T}{2}]\\
f_T(x \pm T) & \text{if } x \notin [-\frac{T}{2},\frac{T}{2}] 
\end{array} \right. \]

\textit{当$T \rightarrow \infty$时,我们可以得到 }:
\[\lim_{T \rightarrow \infty}{f_T(x)=f(x)}\]

\textit{则,由$(g)$式可知,对于一个可以展开成傅里叶级数且其周期$T$无穷大的函数,其傅里叶级数可以写成 }:
\[f(x)=\lim_{T \rightarrow \infty}\frac{1}{T}\sum_{n=-\infty}^{+\infty}(\int_{-\frac{T}{2}}^{\frac{T}{2}}f(y)e^{-i\omega_n y}\dd{y})e^{i\omega_n x} \tag{h}\]

\textit{这里,我们定义一个新的量$\Delta{\omega}$ }:
\[\Delta{\omega}:=\omega_n-\omega_{n-1}=\omega=\frac{2\pi}{T}\]

\textit{则$(h)$式可以写成 }:
\[f(x)=\lim_{\Delta{\omega} \rightarrow 0}\frac{1}{2\pi}\sum_{n=-\infty}^{+\infty}(\int_{-\frac{T}{2}}^{\frac{T}{2}}f(y)e^{-i\omega_n y}\dd{y})e^{i\omega_n x}\Delta{\omega} \tag{i}\]

\textit{由积分的定义式我们可以知道$(i)$式最终可以写成 }:
\[f(x)=\frac{1}{2\pi}\int_{-\infty}^{+\infty}(\int_{-\infty}^{+\infty}f(y)e^{-i\omega y}\dd{y})e^{i\omega x}\dd{\omega} \tag{j}\]

\textit{上面我们不严格地推导处理傅里叶积分公式,下面我们给出傅里叶积分公式的严格定义 }:

\textbf{傅里叶积分定理}

\text{若$f(x)$在${(-\infty,+\infty)}$满足:}

\qquad \qquad \text{在任一有限区间上满足迪利克雷条件;}

\qquad \qquad \text{在${(-\infty,+\infty)}$上绝对可积(即$\int_{-\infty}^{+\infty}|f(x)|\dd{x}$收敛);}

\text{当$f(x)$在点$x=x_0$处连续时,则有:}
\[\frac{1}{2\pi}\int_{-\infty}^{+\infty}(\int_{-\infty}^{+\infty}f(y)e^{-i\omega y}\dd{y})e^{i\omega x_0}\dd{\omega}=f(x_0)\]

\text{当$f(x)$在点$x=x_0$处不连续时,则有:}
\[\frac{1}{2\pi}\int_{-\infty}^{+\infty}(\int_{-\infty}^{+\infty}f(y)e^{-i\omega y}\dd{y})e^{i\omega x_0}\dd{\omega}=\frac{1}{2}\left ( \lim_{x \rightarrow x_0^-}{f(x)}+\lim_{x \rightarrow x_0^+}{f(x)} \right )\]

\subsection{傅里叶变换的概念}

\textbf{定义} \text{如果$f(t)$满足傅里叶积分定理,设} :
\[F(\omega)=\int_{-\infty}^{+\infty}f(t)e^{-i\omega t}\dd{t} \tag{1}\] 
\[f(t)=\frac{1}{2\pi}\int_{-\infty}^{+\infty}F(\omega)e^{i\omega t}\dd{\omega} \tag{2}\]

\text{则称$F(\omega)$为$f(t)$的象函数,称(1)式为$f(t)$的傅里叶变换式,记为:}
\[F(\omega)=\mathcal{F}[f(t)]\]

\text{则称$f(t)$为$F(\omega)$的象原函数,称(2)式为$F(\omega)$的傅里叶逆变换式,记为:}
\[f(t)=\mathcal{F}^{-1}[F(\omega)]\]

\subsection{δ函数及其傅里叶变换}

\textit{单位脉冲函数是英国物理学家狄拉克(Dirac)在20世纪20年代引人的,用于描述瞬间或空间几何点上的物理量。例如,瞬时的冲击力、脉冲电流或电压等急速变化的物理量,以及质点的质量分布、点电荷的电量分布等在空间或时间上高度集中的物理量。脉冲函数也称$\delta$函数。}

\textbf{定义} \qquad \text{若在一维空间中,自变量为时间t的函数$\delta(t)$,满足下述两个条件:}
\[\delta(t)=\left\{
\begin{array}{rl}
0 & \text{if } x \neq 0\\
1 & \text{if } x=0
\end{array} \right. \tag{1}\]
\[\int_{-\infty}^{+\infty}\delta(t)\dd{t}=1 \tag{2}\]

\textit{不论是$\delta$函数还是有延时的$\delta$函数$(\delta(t-t_0))$,都有如下重要的性质} (\textbf{筛选性质}) :

\[\int_{-\infty}^{+\infty}\delta(t-t_0)f(t)\dd{t}=f(t_0) \tag{k}\]

\textit{利用这个性质,我们可以对单位脉冲函数求其傅里叶变换及傅里叶逆变换} :

\[F(\omega)=\mathcal{F}[\delta(t-t_0)]=\int_{-\infty}^{+\infty}\delta(t-t_0)e^{-i\omega t}\dd{t}=e^{-i\omega t_0}\]

\textit{令$F(\omega)=2\pi \delta(\omega-\omega_0)$则有} : 
\[f(t)=\mathcal{F}^{-1}[2\pi \delta(\omega-\omega_0)]=\frac{1}{2\pi}\int_{-\infty}^{+\infty}2\pi \delta(\omega-\omega_0)e^{i\omega t}\dd{\omega}=e^{i\omega_0 t}\]

\textit{这是我们得到两对傅里叶变换对} :

\[\delta(t-t_0) \ and \ e^{-i\omega t_0} \quad,\quad e^{i\omega_0 t} \ and \ 2\pi \delta(\omega-\omega_0) \tag{l}\]

\textit{特别的,我们令$t_0=0,\omega_0=0$,则有以下两个重要的傅里叶变换对} :

\[\delta(t) \ and \ 1 \quad,\quad 1 \ and \ 2\pi \delta(\omega) \tag{m}\]

\textit{此外$\delta$函数还有另一个性质} :

\textbf{定义} \qquad \textit{设函数$u(t)$为单位阶跃函数,即} :

\[u(t)=\left\{ \begin{array}{rl}
1 & \text{if } x>0 \\
0 & \text{if } x<0 
\end{array} \right. \]

\textit{则我们可以获得以下关系式 }:

\[\frac{\mathrm{d}}{\mathrm{d} t}u(t)=\delta(t)\]

\[F(\omega)=\mathcal{F}[u(t)]=\int_{-\infty}^{+\infty}\frac{1+sgn(t)}{2}e^{-i\omega t}\dd{t}=\frac{1}{i\omega}+\pi\delta(\omega)\]

\[u(t)=\mathcal{F}^{-1}[F(\omega)]=\frac{1}{2\pi}\int_{-\infty}^{+\infty}\Bigl(\frac{1}{i\omega}+\pi\delta(\omega) \Bigl)e^{i\omega t}\dd{t}=\frac{1}{2}+\frac{1}{2\pi}\int_{-\infty}^{+\infty}\frac{\sin\omega}{\omega}\dd{\omega} \tag{n}\]

\textit{$\delta$函数并不是严格意义上数域到数域的函数,而是个定义在函数空间上的泛函,它的引入扩充了傅里叶变换的适用范围,使得一些不满足条件$\int_{-\infty}^{+\infty}|f(x)|\dd{x}<+\infty$的函数,如常值函数、符号函数、单位阶跃函数、正弦函数、余弦函数都能进行广义傅里叶变换.}

\textit{例如下面两个例子以及上面的常值函数以及单位阶跃函数 }:

\[\mathcal{F}[\cos(\omega_0t)]=\pi[\delta(\omega+\omega_0)+\delta(\omega-\omega_0)] \tag{o1}\]

\[\mathcal{F}[\sin(\omega_0t)]=i\pi[\delta(\omega+\omega_0)-\delta(\omega-\omega_0)] \tag{o2}\]

\subsection{傅里叶变换的性质}

\textit{这一小节我们介绍傅里叶变换的几个性质,为方便起见,我们规定在该小节中处理的函数均满足傅里叶积分定理的条件。}

\text{1.线性性质}

\[\mathcal{F}[af_1(t)+bf_2(t)]=a\mathcal{F}[f_1(t)]+b\mathcal{F}[f_2(t)]\]
\[\mathcal{F}^{-1}[af_1(t)+bf_2(t)]=a\mathcal{F}^{-1}[f_1(t)]+b\mathcal{F}^{-1}[f_2(t)]\]
\[(a,b \in \mathcal{R})\]

\text{2.位移性质}
\[\mathcal{F}[f(t \pm t_0)]=e^{\pm i\omega t_0}\mathcal{F}[f(t)]\]
\[\mathcal{F}^{-1}[f(t \pm t_0)]=e^{\mp i\omega t_0}\mathcal{F}^{-1}[f(t)]\]

\text{3.微分性质}

\textit{若$f^{(k)}(x)$在$(-\infty,+\infty)$$(k=0,1,2,...,n-1)$连续或者只有有限个间断点,且满足 }
\[\lim_{|t|\rightarrow +\infty}f^{(k)}(t)=0\]

\textit{则有 }:
\[\mathcal{F}[f^{(n)}(t)]=(i\omega)^n \mathcal{F}[f(t)]\]
\[\frac{\mathrm {d}^n}{\mathrm {d} \omega^n} \mathcal{F}[f(t)]=(-i)^n \mathcal{F}[t^nf(t)]\]

\text{4.积分性质}

\textit{当$t \rightarrow +\infty$时,$\int_{-\infty}^{t}f(t)\dd{t} \rightarrow 0$,则 }:
\[ \mathcal{F}[\int_{-\infty}^{t}f(t)\dd{t}]=\frac{1}{i\omega}\mathcal{F}[f(t)]\]

\text{5.乘积定理}

\textit{若$F_1(\omega)=\mathcal{F}[f_1(t)]$,$F_2(\omega)=\mathcal{F}[f_2(t)]$,则 }:

\[\int_{-\infty}^{+\infty}\overline{f_1(t)}f_2(t)\dd{t}=\frac{1}{2\pi}\int_{-\infty}^{+\infty}\overline{F_1(\omega)}F_2(\omega)\dd{\omega}\]
\[\int_{-\infty}^{+\infty}f_1(t)\overline{f_2(t)}\dd{t}=\frac{1}{2\pi}\int_{-\infty}^{+\infty}F_1(\omega)\overline{F_2(\omega)}\dd{\omega}\]

\text{6.能量积分}

\textbf{Parseval等式} \qquad \textit{若$F(\omega)=\mathcal{F}[f(t)]$,则 }:

\[\int_{-\infty}^{+\infty}[f(t)]^2\dd{t}=\frac{1}{2\pi}\int_{-\infty}^{+\infty}[F(\omega)]^2\dd{\omega}:=\frac{1}{2\pi}\int_{-\infty}^{+\infty}S(\omega)\dd{\omega}\]

\subsection{卷积与卷积定理}

\textit{若已知函数$f_1(t)$、$f_2(t)$,则我们定义卷积为 }:

\[f_1(t)\ast f_2(t):=\int_{-\infty}^{+\infty}f_1(\tau)f_2(t-\tau)\dd{\tau} \tag{p}\] 

\textit{容易验证卷积运算满足 }:

\textit{1.交换律:$f_1(t)\ast f_2(t)=f_2(t)\ast f_1(t)$}

\textit{2.结合律:$(f_1(t) \ast f_2(t)) \ast f_3(t)=f_1(t) \ast (f_2(t) \ast f_3(t))$}

\textit{3.对加法的分配律:$f_1(t) \ast (f_2(t)+f_3(t))=f_1(t) \ast f_2(t)+f_1(t) \ast f_3(t)$}

\textit{此外,卷积运算还满足以下性质 } :

\textit{卷积的数乘 }:

\[a[f_1(t)\ast f_2(t)]=[af_1(t)]\ast f_2(t)=f_1(t)\ast [af_2(t)] \qquad (a \in \mathcal{R})\]

\textit{卷积的微分 }:

\[\frac{\mathrm {d}}{\mathrm {d}x}[f_1(t)\ast f_2(t)]=[\frac{\mathrm {d}}{\mathrm {d}x}f_1(t)]\ast f_2(t)=f_1(t)\ast\frac{\mathrm {d}}{\mathrm {d}x}[ f_2(t)]\]

\textit{卷积不等式 }:

\[|f_1(t)\ast f_2(t)| \leqslant |f_1(t)|\ast |f_2(t)|\]

\textit{与$\delta$函数相关的性质 }:

\[f(t) \ast \delta(t)=f(t)\]

\textbf{卷积定理} \qquad \textit{若$f_1(t)$和$f_2(t)$满足傅里叶积分定理的条件,则有 }:

\[\mathcal{F}[f_1(t) \ast f_2(t)]=\mathcal{F}[f_1(t)] \cdot \mathcal{F}[f_2(t)]\]

\subsection{相关函数}

\textit{给定两函数$f_1(t)$、$f_2(t)$,则定义互相关函数 }:

\[R_{12}(\tau):=\int_{-\infty}^{+\infty}f_1(t)f_2(t+\tau)\dd{t} \qquad R_{21}(\tau):=\int_{-\infty}^{+\infty}f_1(t+\tau)f_2(t)\dd{t}\]

\textit{当$f_1(t)=f_2(t)$时,则定义自相关函数 }:

\[R(\tau):=\int_{-\infty}^{+\infty}f(t)f(t+\tau)\dd{t}\]

\textit{关于互相关函数有如下性质 }:

\[R_{21}(\tau)=R_{12}(-\tau)\]

\textit{若令乘积定理中实函数$f_1(t)=f_1(t)$,$f_2(t)=f_2(t+\tau)$,则有 }:

\[\int_{-\infty}^{+\infty}f_1(t)f_2(t+\tau)\dd{t}=\frac{1}{2\pi}\int_{-\infty}^{+\infty}\overline{F_1(\omega)}F_2(\omega)e^{i\tau \omega}\dd{\omega} :=\frac{1}{2\pi}\int_{-\infty}^{+\infty}S_{12}(\omega)e^{i\tau \omega}\dd{\omega}\]

\[R_{12}(\tau)=\frac{1}{2\pi}\int_{-\infty}^{+\infty}S_{12}(\omega)e^{i\tau \omega}\dd{\omega} \qquad S_{12}(\omega)=\overline{S_{21}(\omega)} \tag{q}\]

\textit{当$f_1(t)=f_2(t)$时} (q) \textit{式可做出相应的简化。}

\section{傅里叶变换的应用}

\subsection{周期函数与离散频谱}

\textit{对一个周期为$T$的函数$f(t)$其傅里叶级数为 }:

\[f(t)=\sum_{n=-\infty}^{+\infty}g(n)e^{in\omega_0 t} \]

\textit{对上式两边同时取傅里叶变换,并认为系数项不为时间$t$的函数,由此可得 }:

\[F(\omega)=\mathcal{F}[f(t)]=\int_{-\infty}^{+\infty}[\sum_{n=-\infty}^{+\infty}g(n)e^{in\omega_0 t}]e^{-i\omega t}\dd{t}=2\pi\sum_{n=-\infty}^{+\infty}g(n)\delta(\omega-n\omega_0) \tag{r}\]

\subsection{非周期函数与连续频谱}

\textit{对非周期函数$f(t)$,若其满足傅里叶积分定理条件,则在$f(t)$的连续点处,有 }:

\[f(t)=\frac{1}{2\pi}\int_{-\infty}^{+\infty}F(\omega)e^{i\omega t}\dd{\omega}\]

\textit{其傅里叶变换式为} :

\[F(\omega)=\int_{-\infty}^{+\infty}f(t)e^{-i\omega t}\dd{t}\] 

\textit{在频谱分析中,傅里叶变换式$F(\omega)$又称为$f(t)$的频谱函数,频谱函数的模$|F(\omega)|$称为$f(t)$的振幅频谱(亦可简称为频谱)。在傅里叶积分中当$n\rightarrow +\infty$时,频率间隔$\Delta \omega $成为$\dd{\omega}$,$\omega$为连续变量,故称$|F(\omega)|$为连续频谱。对一个时间函数$f(t)$做傅里叶变换,就是求$f(t)$的频谱。}