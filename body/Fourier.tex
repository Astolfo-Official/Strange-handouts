\section{傅里叶变换的部分相关概念及性质}

\subsection{傅里叶级数}
利用一套完备的正交函数系,我们可以将一个周期函数展开成该正交函数系的线性组合。对于周期函数来说,最常用的完备正交函数系就是三角函数系$\{\cos(n\omega x),\sin(n\omega x)\}$。下面我们给出傅里叶级数定理:
\begin{theorem}[傅里叶级数定理]
当一个周期函数$f(x)$满足迪利克雷(Dirichlet)条件,即在区间${[-\frac{T}{2},\frac{T}{2}]}$上满足
\begin{itemize}
    \item 1. f(x)在区间${[-\frac{T}{2},\frac{T}{2}]}$上连续或者只有有限个第一间断点
    \item 2. f(x)在区间${[-\frac{T}{2},\frac{T}{2}]}$上只有有限个极值点
\end{itemize}
$f(x)$就可以展开成傅里叶级数
\[f(x)=\frac{a_0}{2}+\sum_{n=1}^{+\infty}(a_n\cos(n\omega x)+b_n\sin(n\omega x))\]
这里我们利用周期$T$定义频率$\omega$
\[\omega=\frac{2\pi}{T}\]
则傅立叶展开的系数可以表示成
\[a_0=\frac{2}{T}\int_{-\frac{T}{2}}^{\frac{T}{2}}f(x)\dd{x}, \quad a_n=\frac{2}{T}\int_{-\frac{T}{2}}^{\frac{T}{2}}f(x)\cos(n\omega x)\dd{x}, \quad b_n=\frac{2}{T}\int_{-\frac{T}{2}}^{\frac{T}{2}}f(x)\sin(n\omega x)\dd{x}\]
\end{theorem}
将$\cos(n\omega x)$、$\sin(n\omega x)$写成复指数形式,则傅里叶级数可以写成以下形式
\[f(x)=\frac{a_0}{2}+\sum_{n=1}^{+\infty}(\frac{a_n-ib_n}{2}e^{in\omega x}+\frac{a_n+ib_n}{2})e^{-in\omega x}\]
定义一套新的系数$\{c_n\}$:
\[c_0=\frac{a_0}{2}=\frac{1}{T}\int_{-\frac{T}{2}}^{\frac{T}{2}}f(x)\dd{x}\]
\[c_n=\frac{a_n-ib_n}{2}=\frac{1}{T}(\int_{-\frac{T}{2}}^{\frac{T}{2}}f(x)(\cos(n\omega x)\dd{x}-i\sin(n\omega x))\dd{x})=\frac{1}{T}\int_{-\frac{T}{2}}^{\frac{T}{2}}f(x)e^{-in\omega x}\dd{x}\]
\[c_{-n}=\frac{a_n+ib_n}{2}=\frac{1}{T}(\int_{-\frac{T}{2}}^{\frac{T}{2}}f(x)(\cos(n\omega x)\dd{x}+i\sin(n\omega x))\dd{x})=\frac{1}{T}\int_{-\frac{T}{2}}^{\frac{T}{2}}f(x)e^{in\omega x}\dd{x}\]
即可得到复指数形式的傅立叶级数定理:
\begin{theorem}[复指数形式的傅里叶级数定理]
\[f(x)=c_0+\sum_{n=1}^{+\infty}(c_ne^{in\omega x}+c_{-n}e^{-in\omega x})=\sum_{n=-\infty}^{+\infty}c_ne^{in\omega x}, \quad c_n=\frac{1}{T}\int_{-\frac{T}{2}}^{\frac{T}{2}}f(x)e^{-in\omega x}\dd{x}\]
或者简单写成
\[f(x)=\sum_{n=-\infty}^{+\infty}(\frac{1}{T}\int_{-\frac{T}{2}}^{\frac{T}{2}}f(y)e^{-i\omega_n y}\dd{y})e^{i\omega_n x}\]
\end{theorem}
对于周期函数,其傅立叶变换会退化会离散的傅立叶级数。
\subsection{傅里叶积分定理与傅里叶变换}
在实际的应用中,我们得到的信号曲线不一定是一个周期函数,那么我们有没有办法将一个非周期的函数展开成傅里叶级数呢?对于一个定义在实数轴上的非周期的函数,我们可以认为它的周期是无穷大,对此我们将把能展开成傅里叶级数的周期函数延拓到整个实数轴上,再将其周期取为无穷大。

将$f(x)$按周期$T$在实数域上做延拓的到函数$f_T(x)$:
\[ f_T(x) := \left\{
\begin{array}{rl}
f(x) & \text{if } x \in [-\frac{T}{2},\frac{T}{2}]\\
f_T(x \pm T) & \text{if } x \notin [-\frac{T}{2},\frac{T}{2}] 
\end{array} \right.\]
把周期扩大到无穷$T \rightarrow \infty$时,我们可以得到:
\[\lim_{T \rightarrow \infty}{f_T(x)=f(x)}\]
对于一个可以展开成傅里叶级数且其周期$T$无穷大的函数,其傅里叶级数可以写成
\[f(x)=\lim_{T \rightarrow \infty}\frac{1}{T}\sum_{n=-\infty}^{+\infty}(\int_{-\frac{T}{2}}^{\frac{T}{2}}f(y)e^{-in\omega y}\dd{y})e^{in\omega x}=\lim_{\omega \rightarrow 0}\frac{\omega}{2\pi}\sum_{n=-\infty}^{+\infty}(\int_{-\frac{T}{2}}^{\frac{T}{2}}f(y)e^{-in\omega y}\dd{y})e^{in\omega x}\]
由积分的定义式上式最终可以得到
\[f(x)=\frac{1}{2\pi}\int_{-\infty}^{+\infty}(\int_{-\infty}^{+\infty}f(y)e^{-i\omega y}\dd{y})e^{i\omega x}\dd{\omega}\]
\begin{theorem}[傅里叶积分公式]
当$f(x)$在${(-\infty,+\infty)}$满足:
\begin{itemize}
    \item 1. 在任一有限区间上满足迪利克雷条件;
    \item 2. 在${(-\infty,+\infty)}$上绝对可积(即$\int_{-\infty}^{+\infty}|f(x)|\dd{x}$收敛);
\end{itemize}
当$f(x)$在点$x=x_0$处连续时
\[\frac{1}{2\pi}\int_{-\infty}^{+\infty}(\int_{-\infty}^{+\infty}f(y)e^{-i\omega y}\dd{y})e^{i\omega x_0}\dd{\omega}=f(x_0)\]
当$f(x)$在点$x=x_0$处不连续时
\[\frac{1}{2\pi}\int_{-\infty}^{+\infty}(\int_{-\infty}^{+\infty}f(y)e^{-i\omega y}\dd{y})e^{i\omega x_0}\dd{\omega}=\frac{1}{2}\left ( \lim_{x \rightarrow x_0^-}{f(x)}+\lim_{x \rightarrow x_0^+}{f(x)} \right )\]
\end{theorem}
\begin{definition}[傅里叶变换与傅里叶逆变换]
设$f(t)$在${(-\infty,+\infty)}$上满足傅里叶积分定理的条件,可定义以下积分变换
\[F(\omega)=\int_{-\infty}^{+\infty}f(t)e^{-i\omega t}\dd{t}, \quad f(t)=\frac{1}{2\pi}\int_{-\infty}^{+\infty}F(\omega)e^{i\omega t}\dd{\omega}\]
称$F(\omega)$为$f(t)$的象函数,称$f(t)$到$F(\omega)$的积分变换为傅里叶变换,,$f(t)$为$F(\omega)$的象原函数,称$F(\omega)$到$f(t)$的积分变换为傅里叶逆变换式,记为$F(\omega)=\mathcal{F}[f(t)], \quad f(t)=\mathcal{F}^{-1}[F(\omega)]$。
\end{definition}
傅立叶变换中一个重要的应用对象是$\delta$函数(单位脉冲函数),是英国物理学家狄拉克(Dirac)在20世纪20年代引人的,用于描述瞬间或空间几何点上的物理量。例如,瞬时的冲击力、脉冲电流或电压等急速变化的物理量,以及质点的质量分布、点电荷的电量分布等在空间或时间上高度集中的物理量。
\begin{definition}[单位脉冲函数]
设函数$\delta(t)$为一维空间中自变量为时间t的函数,若它满足:
\[\delta(t)=\left\{
\begin{array}{rl}0 & \text{if } t \neq 0\\
1 & \text{if } t=0
\end{array} \right., \quad \int_{-\infty}^{+\infty}\delta(t)\dd{t}=1\]
则称$\delta(t)$为单位脉冲函数,或简称为$\delta$函数。
\end{definition}
$\delta$函数有如下重要的性质(\textbf{筛选性质}):
\[\int_{-\infty}^{+\infty}\delta(t-t_0)f(t)\dd{t}=f(t_0)\]
利用这个性质,我们可以对单位脉冲函数求其傅里叶变换及傅里叶逆变换:
\[F(\omega)=\mathcal{F}[\delta(t-t_0)]=\int_{-\infty}^{+\infty}\delta(t-t_0)e^{-i\omega t}\dd{t}=e^{-i\omega t_0}\]
令$F(\omega)=2\pi \delta(\omega-\omega_0)$则有: 
\[f(t)=\mathcal{F}^{-1}[2\pi \delta(\omega-\omega_0)]=\frac{1}{2\pi}\int_{-\infty}^{+\infty}2\pi \delta(\omega-\omega_0)e^{i\omega t}\dd{\omega}=e^{i\omega_0 t}\]
这是我们得到两对傅里叶变换对:$\delta(t-t_0)$和$e^{-i\omega t_0}$,$e^{i\omega_0 t}$和$2\pi \delta(\omega-\omega_0)$。可以看出$\delta$函数与平面波是一堆傅立叶变换对。特别的,我们令$t_0=0,\omega_0=0$,则有$\delta(t)$和$1$,$1$和$2\pi \delta(\omega)$。
稍作推广我们可以得到
\[\mathcal{F}[\cos(\omega_0t)]=\pi[\delta(\omega+\omega_0)+\delta(\omega-\omega_0)], \quad \mathcal{F}[\sin(\omega_0t)]=i\pi[\delta(\omega+\omega_0)-\delta(\omega-\omega_0)]\]
\begin{lemma}[傅里叶变换的基本性质]
设$a,b \in \mathbb{C}$,$f,f_1,f_2$是满足傅立叶积分定理的复值函数,则傅立叶变换具有以下性质:
\begin{itemize}
\item 1. 线性
\[\mathcal{F}[af_1(t)+bf_2(t)]=a\mathcal{F}[f_1(t)]+b\mathcal{F}[f_2(t)], \quad \mathcal{F}^{-1}[af_1(t)+bf_2(t)]=a\mathcal{F}^{-1}[f_1(t)]+b\mathcal{F}^{-1}[f_2(t)]\]
\item 2. 位移性质
\[\mathcal{F}[f(t \pm t_0)]=e^{\pm i\omega t_0}\mathcal{F}[f(t)], \quad \mathcal{F}^{-1}[f(t \pm t_0)]=e^{\mp i\omega t_0}\mathcal{F}^{-1}[f(t)]\]
\item 3. 微分性质:若$f \in C^n(-\infty,+\infty)$(可以推广到有限个间断点),且$\lim_{|t|\rightarrow +\infty}f^{(k)}(t)=0$:
\[\mathcal{F}[f^{(n)}(t)]=(i\omega)^n \mathcal{F}[f(t)], \quad \frac{\mathrm {d}^n}{\mathrm {d} \omega^n} \mathcal{F}[f(t)]=(-i)^n \mathcal{F}[t^nf(t)]\]
\item 4. 积分性质:当$t \rightarrow +\infty$时,$\int_{-\infty}^{t}f(t)\dd{t}<+\infty$,则
\[ \mathcal{F}[\int_{-\infty}^{t}f(t)\dd{t}]=\frac{1}{i\omega}\mathcal{F}[f(t)]+\pi\delta(\omega)\int_{-\infty}^{\infty}f(t)\dd{t}\]
\item 5. 乘积定理:若$F_1(\omega)=\mathcal{F}[f_1(t)]$,$F_2(\omega)=\mathcal{F}[f_2(t)]$,则
\[\int_{-\infty}^{+\infty}f_1^{\dagger}(t)f_2(t)\dd{t}=\frac{1}{2\pi}\int_{-\infty}^{+\infty}F_1^{\dagger}(\omega)F_2(\omega)\dd{\omega}, \quad \int_{-\infty}^{+\infty}f_1(t)f_2^{\dagger}(t)\dd{t}=\frac{1}{2\pi}\int_{-\infty}^{+\infty}F_1(\omega)F_2^{\dagger}(\omega)\dd{\omega}\]
\item 6.能量积分(Parseval等式)
\[\int_{-\infty}^{+\infty}[f(t)]^2\dd{t}=\frac{1}{2\pi}\int_{-\infty}^{+\infty}[F(\omega)]^2\dd{\omega}:=\frac{1}{2\pi}\int_{-\infty}^{+\infty}S(\omega)\dd{\omega}\]
\end{itemize}
\end{lemma}
\subsection{卷积与相关函数}
我们可以定义卷积运算及相关函数
\begin{definition}[卷积]
若已知函数$f_1(t),f_2(t)$,则定义卷积为
\[f_1(t)\ast f_2(t):=\int_{-\infty}^{+\infty}f_1(\tau)f_2(t-\tau)\dd{\tau}\] 
\end{definition}
\begin{definition}[相关函数]
给定两函数$f_1(t)$、$f_2(t)$,定义相关函数($f_1(t)=f_2(t)$时,称$R(\tau)$为自相关函数):
\[R(\tau):=\int_{-\infty}^{+\infty}f_1(t)f_2(t+\tau)\dd{t}\]
\end{definition}
\begin{lemma}[卷积的性质]
\begin{itemize}
\item 1. 交换律:$f_1(t)\ast f_2(t)=f_2(t)\ast f_1(t)$
\item 2. 结合律:$(f_1(t) \ast f_2(t)) \ast f_3(t)=f_1(t) \ast (f_2(t) \ast f_3(t))$
\item 3. 线性性:$f_1(t) \ast (af_2(t)+bf_3(t))=af_1(t) \ast f_2(t)+f_1(t) \ast bf_3(t),  \ a,b\in\mathbb{C}$
\item 4. 微分:
\[\frac{\mathrm {d}}{\mathrm {d}x}[f_1(t)\ast f_2(t)]=[\frac{\mathrm {d}}{\mathrm {d}x}f_1(t)]\ast f_2(t)=f_1(t)\ast\frac{\mathrm {d}}{\mathrm {d}x}[ f_2(t)]\]
\item 5. 卷积不等式:
\[|f_1(t)\ast f_2(t)| \leqslant |f_1(t)|\ast |f_2(t)|\]
\item 6. $\delta$函数的卷积性质:
\[f(t) \ast \delta(t)=f(t)\]
\end{itemize}
\end{lemma}
卷积最重要的性质是卷积定理
\begin{theorem}[卷积定理]
    若$f_1(t)$和$f_2(t)$满足傅里叶积分定理的条件,则有
\[\mathcal{F}[f_1(t) \ast f_2(t)]=\mathcal{F}[f_1(t)] \cdot \mathcal{F}[f_2(t)]\]
\end{theorem}

\section{傅里叶变换的应用}
