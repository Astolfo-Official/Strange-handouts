\section{如何理解原子、分子上电子运动范围和能量的关系}
运动范围可以考虑成以下两种情况,一是最概然分布位置,二是边界束缚。

首先考虑最概然分布位置,最概然分布的情况对应于波函数的收敛性质不依赖于边界条件,如氢原子中的轨道等,在这种情况下主量子数决定了轨道的最概然分布位置$<r>$,主量子数越大,$<r>$越大,$<V>$势能越大,考虑位力定理,$2<T>=n<V>$,这里取$n=-1$,体系能量的均值$<E>=-\frac{1}{2}<V>$越大。

其次考虑边界束缚,边界束缚的条件对应于波函数收敛性质与边界条件有关,如势箱,在这种情况下,电子的运动范围越广,能量越低。
